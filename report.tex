\documentclass{article}
\usepackage[utf8]{inputenc}
\usepackage{geometry}
 \geometry{
 a4paper,
 total={170mm,257mm},
 left=20mm,
 top=20mm,
 }
 \usepackage{graphicx}
 \usepackage{titling}

 \title{Simulating capital return and income growth to explore their impact on wealth inequality
}
\author{Haoran Jie hj376}
\date{January 2023}
 
 \usepackage{fancyhdr}
\fancypagestyle{plain}{%  the preset of fancyhdr 
    \fancyhead[L]{Tick4}
    \fancyhead[R]{\theauthor}
}
\makeatletter
\def\@maketitle{%
  \newpage
  \null
  \vskip 1em%
  \begin{center}%
  \let \footnote \thanks
    {\LARGE \@title \par}%
    \vskip 1em%
    %{\large \@date}%
  \end{center}%
  \par
  \vskip 1em}
\makeatother

\usepackage{lipsum}  
\usepackage{cmbright}

\begin{document}

\maketitle


\section*{Goals}
Argued by the economist Thomas Piketty in his book \textit{Capital in the Twenty-first Century}, the growth of income from capital (i.e. investments, property, etc.)\footnote{In this study we simplify the situation by assuming capital to be the current wealth} has been outpacing economic growth, leading to increased inequality. This report gather, simulate, and analyse data to test the validity of Piketty's theory and to understand the causes and implications of increasing inequality.

\section*{Methodology}
1000 individuals randomly assigned per-timestep income
Return on capital growth factor
Gini coeficient to measure inequality
Also measure economic mobility because "some degree of inequality might be acceptible if economic mobility were high"
Incorporate exchange model to allow for certain level of uncertainty (The extension for tick 1)\\
1. Choose a return on capital smaller than the economic growth and see how the inequality changes overtime\\
2. Choose a return on capital higher than the economic growth and see how the inequality changes over time\\
Normalize: keep the sum of wealth of the 1000 people at 1000 if it exceeds 1000
\section*{Result}


\section*{Conclusion}


\end{document}
